\documentclass[25pt]{book}
\usepackage{amsmath}
\usepackage{fullpage}
\usepackage{graphicx}
\usepackage{ gensymb }
\usepackage{wasysym}
\usepackage{multicol}
\usepackage{braket}
\usepackage{dsfont}
\usepackage{url}
\usepackage{bm}
\usepackage{float}
\usepackage[margin=1in]{geometry}
\usepackage{bookmark}
\usepackage[colorlinks=true,linkcolor=blue,urlcolor=black,bookmarksopen=true]{hyperref}

\begin{document}
	
\tableofcontents

\chapter{Problem Types}

\chapter{Derivations}

\chapter{Concepts}

\chapter{Equations}

\section{Euler-Lagrange Eq.'s}

Euler-Lagrange Equations

For one coordinate $q$
\[
\frac{d}{dt} \frac{\partial \mathcal{L}}{\partial \dot{q}} - \frac{\partial \mathcal{L}}{\partial q}= 0
\]

\section{Invariance of the Lagrangian}

For the two Lagrangians 
\[
\mathcal{L} = T - V
\]
and 
\[
\mathcal{L}' = T - V + \frac{d f(x,t)}{dt}
\]
the dynamics are exactly the same for \textit{any} function $f(x,t)$. 

\section{Parallel Axis Theorem}
\textbf{Also called Steiner's Theorem}

Given the moment of inertia about the center of mass, this theorem allows us to calculate the moment of inertia about an axis offset from the center (although still pointing in the same direction). For center-of-mass MoI $I_c$, mass $M$, and axis offset $h$, the new moment of inertia is
\[
I = I_c + Mh^2
\]

\section{Hamilton-Jacobi Equation}

For the Hamiltonian $\mathcal{H}$
\[
H
\]

\section{Hamilton's Eq.'s}



\end{document}
